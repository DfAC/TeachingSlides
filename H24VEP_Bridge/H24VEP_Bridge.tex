
%TeXstudio produce a LOT of warning yet still compile.

\documentclass[11pt]{beamer} 

\usetheme{metropolis} %https://github.com/matze/mtheme
\metroset{everytitleformat = regular,progressbar=foot} %settings
\mode<presentation>
%\usecolortheme{dove} %dove
% albatross, beaver, beetle, crane, default, dolphin, dove, orchid, rose, seagull, seahorse, whale, wolverine
%dont use  fly, lily,
%http://mirror.ox.ac.uk/sites/ctan.org/macros/latex/contrib/beamer/doc/beameruserguide.pdf
\setbeamercolor{title separator}{fg = UniBlue}
\setbeamercolor{frametitle}{fg = deepBlue, bg=aBlue!70}

\usepackage{booktabs}
\usepackage[scale=2]{ccicons}
\usepackage{pgfplots}
\usepgfplotslibrary{dateplot}
%\usepackage[demo]{graphicx}
\usepackage[font={small,it}]{caption}
\usepackage{subcaption} %throws a lot of errors but compile

%%LOGO
\usepackage{tikz}
\addtobeamertemplate{frametitle}{}{%
\begin{tikzpicture}[remember picture,overlay]
\node[anchor=north east,yshift=2pt] at (current page.north east) {\includegraphics[height=0.8cm]{./../logo.png}};
\end{tikzpicture}}

%My std preamble for the docs
%\selectlanguage{british}%
\usepackage[british]{babel}
\usepackage{microtype} %better text
\IfFileExists{lmodern.sty}{\usepackage{lmodern}}{} %type 1 vector font
%
\usepackage{lettrine}
\usepackage{listings} %Add list support
\usepackage{colortbl} %colors in TABLES
%\usepackage{tikz,amsmath, amssymb,bm,color}
\usepackage{nicefrac}
\usepackage{lastpage} %get last page

%COLORS
\usepackage{color}
\definecolor{lightgray}{gray}{0.8} %for colortbl


%%%%%%%%%%%%%%%%%%%%%%%%%%%%%%%%%%%%%%%%%%%%%%%%%%%%%%%%%%%%%%%%%%
%FOOTNOTES
%nice look after http://www.dedoimedo.com
\usepackage[bottom,hang,splitrule]{footmisc} %\usepackage[bottom]{footmisc}
\setlength{\footnotemargin}{0.35cm} %space between number and footnote text
\setlength{\footnotesep}{0.2cm} %{12pt} %space between footnotes
\setlength{\skip\footins}{0.3cm} % space between the text body and the footnotes
%\setlength{\footskip}{0pt} dont do it. It will overflow page no

%%%%%%%%%%%%%%%%%%%%%%%%%%%%%%%%%%%%%%%%%%%%%%%%%%%%%%%%%%%%%%%%%%%
%TABLE SETTINGS
\usepackage{colortbl} %colors in table
\usepackage{rotating} %rotatins within tables
\usepackage{multirow}
\renewcommand{\arraystretch}{1.2} %add padding/spacing
%\usepackage{adjustbox}%rotating and fitting into page
%

\usepackage{booktabs} % To thicken table lines
%define thickness of table lines
\let\mytoprule\toprule
\renewcommand{\toprule}{\mytoprule[0.20em]}
\let\mytoprule\bottomrule
\renewcommand{\bottomrule}{\mytoprule[0.20em]}
\let\mytoprule\midrule
\renewcommand{\midrule}{\mytoprule[0.08em]}

\usepackage{spreadtab} % for simple calculations

%vertically and horizontally centered multicolumn cells with a fixed width. M{width}
%\newcolumntype{M}[1]{>{\centering\hspace{0pt}}m{#1}}
% each spanned cell has the same width. S{width of multicolumn cell}{number of spanned columns}
%\newcolumntype{S}[2]{>{\centering\hspace{0pt}}m{(#1+(2\tabcolsep+\arrayrulewidth)*(1-#2))/#2}}

%%%%%%%%%%%%%%%%%%%%%%%%%%%%%%%%%%%%%%%%%%%%%%%%%%%%%%%%%%%%%%%%%%%
%TikZ
\usepackage{tikz}

\colorlet{red}{red!50}
\colorlet{green}{green!50}
\colorlet{blue}{blue!50}
\definecolor{yellow}{HTML}{FFFF00}
\colorlet{yellow}{yellow!50}
\definecolor{fiolet}{HTML}{7030A0}
\colorlet{fiolet}{fiolet!50}
\colorlet{bgd_main}{black!50}
\colorlet{bgd}{bgd_main!75}
\colorlet{bgd2}{bgd_main!50}
\colorlet{bgd3}{bgd_main!25}
\colorlet{bgd4}{bgd_main!15}

\usetikzlibrary{shapes,arrows,calc,positioning}

%%%%%%%%%%%%%%%%%%%%%%%%%%%%%%%%%%%%%%%%%%%%%%%%%%%%%%%%%%%%%%%%%%%
% Footnotes in Figs
%\rule[raise-height]{width}{thickness}
\newcommand*{\FigFootnote}[1]{
\noindent \begin{flushleft}
\rule[0.2ex]{0.4\columnwidth}{0.5pt}
\par
\footnotesize
#1
\footnotesize
\end{flushleft}
}

%%%%%%%%%%%%%%%%%%%%%%%%%%%%%%%%%%%%%%%%%%%%%%%%%%%%%%%%%%%%%%%%%%%
%MATH, number display
%need to install siunitx, l3kernel,l3packages
\usepackage{siunitx} %this is for units display

\sisetup{per-mode=fraction, tight-spacing = true , fraction-function = \nicefrac, quotient-mode = fraction}% %nicefrac \tfrac
\sisetup{inter-unit-product = \ensuremath { { } \cdot { } } , exponent-product = \cdot }%
\sisetup{input-product=x , output-quotient =  \ensuremath { { } \times{}}} %for 1x2x3
%number grouping(3), std==true %\sisetup{group-digits = decimal} 
\sisetup{group-minimum-digits = 4} %start grouping from 4 digits, in 3 no groups
\sisetup{range-units = single,range-phrase = \,--\,} %2-3C not 2C-3C %, range-phrase = --
\sisetup{separate-uncertainty=true} %2+-1 not 2(1)
\sisetup{prefixes-as-symbols=true } % , scientific-notation = engineering false for 10^-9 ect ect , exp in multiple of 3
\sisetup{range-phrase = \,-\, } % , refo of ranges
%\sisetup{zero-decimal-to-integer, round-mode = places,round-precision = 3}
%\sisetup{add-arc-degree-zero=true , add-arc-minute-zero=true ,add-arc-second-zero=true} %for angle settings

%This is to auto convert ns,ms,us to 10^-xx s
\DeclareSIUnit[scientific-notation = engineering, prefixes-as-symbols=false]{\psec}{\pico\second}
\DeclareSIUnit[scientific-notation = engineering, prefixes-as-symbols=false]{\nsec}{\nano\second} 
\DeclareSIUnit[scientific-notation = engineering, prefixes-as-symbols=false]{\usec}{\micro\second}
\DeclareSIUnit[scientific-notation = engineering, prefixes-as-symbols=false]{\msec}{\milli\second} 

%...AND SOME UNITS
\DeclareSIUnit\dBm{dBm}
\DeclareSIUnit\ppm{ppm} %{\num{1e-6}}%{ppm}
\DeclareSIUnit\yr{yr} %{{361}\day}<-nice nice
\DeclareSIUnit\cy{cycle} %phase cycle
\DeclareSIUnit\epoch{epoch} %GPS/LL epoch
\DeclareSIUnit\inch{"} %inch 
\DeclareSIUnit\wk{week} %week
\DeclareSIUnit\hr{hrs} %hours
\DeclareSIUnit\min{minute} %hours
\DeclareSIUnit\mile{mi}
\DeclareSIUnit\Mcps{Mcps}
\DeclareSIUnit\bit{bit}
\DeclareSIUnit\chip{chip length}
%Other units
\newcommand*{\GBP}[1]{$\SI{#1}[\textsterling]{}$}


%%%%%SHORTHANDS (Standard Sentences)
%\newcommand*{\Myrange}[3]{$\textrm{\SIrange{#1}{#2}{#3}}$}

%details of bar tracking name
\newcommand*{\bsSys}{\textit{PASS Locator booking system}\xspace}


%how much work per week
\newcommand*{\wkWrk}[1]{$\SI{#1}{\hr\per\wk}$}

%references
\newcommand*{\tabref}[1]{shown in table \ref{#1} on page \pageref{#1}\xspace}
\newcommand*{\vref}[1]{\ref{#1} on page \pageref{#1}\xspace}


%\renewcommand{\baselinestretch}{1.2} %line spacing
%{\setstretch{1.0}\color{blue} text bla bla } for section strech
\renewcommand{\footnotesize}{\scriptsize} %change footnote sizes
\newcommand{\MatlabRef}{\footnote{History of changes at \url{https://github.com/DfAC/TEQCSPEC}.}}
\newcommand{\thisDocRef}{\footnote{History of changes at \url{https://github.com/DfAC/TeachingSlides/}.}}
%\usepackage{lipsum} % for dummy text only

%Grey text
\definecolor{shadecolor}{RGB}{190,190,190}
%\textcolor{shadecolor}{	\item[Ogaja\_Matlab.7z] Matlab script. To be placed on-line after week2.}


%%%%%%%%%%%%%%%%%%%%%%%%%%%%%%%%%%%%%%%%%%%%%%%%%%%%%%%%%%%%%%%%%%%%%%%%%%%%%%
%%%%%%%%%%%%%%%%%%%%%%%%%%%%%%%%%%%%%%%%%%%%%%%%%%%%%%%%%%%%%%%%%%%%%%%%%%%%%%
\title[H24VEP]{H24VEP Project 3}
\subtitle{Bridge Monitoring}

\author{LKB/SI}
\institute{NGI}
\date{\today}

\begin{document}
	
	\begin{frame}
		\titlepage
	\end{frame}
	
	% Uncomment these lines for an automatically generated outline.
	\begin{frame}{Outline}
		\tableofcontents
	\end{frame}
	
\section{Introduction}

\begin{frame}[allowframebreaks]{Project Introduction}

The aim of the project is to characterise bridge movement using precise geodetic sensors. We will monitor Wilford Pedestrian Bridge Monitoring Using GNSS, Robotic Total Station (RTS), IMU and Inclinometer sensors. We will also try to excite the structure to better understand sensors performance and to characterise bridge movement.

To do so we will:

\begin{itemize}
	\item Identify the best layout of the sensors on the bridge;
	\item Decide the details of the trial next week;
	\item Analyse measurements made through different sensors;
	\item Examine the bridge's response under different excitation types;
	\item Compare and contrast results;
	\item Find the type of excitation which best shows the differences between equipment outputs.
\end{itemize}
\end{frame}

\begin{frame}{Wilford Pedestrian Bridge}
Wilford Pedestrian Bridge is single span suspension bridge. Spans the River Trent; 69 m long and 3.7 m wide.

	\includegraphics[height=.4\textheight]{pic/Bridge01.jpg}
	\captionof{figure}{Wilford Pedestrian Bridge} %to make images on top of each other
		\centering
	\includegraphics[width=.4\textwidth]{pic/Bridge02.jpg} 
	%\includegraphics[width=.4\textwidth]{pic/Bridge02.jpg} 

\end{frame}

\begin{frame}{Equipment}

We plan to use following equipment on the bridge:

\begin{itemize}
	\item Leica GNSS receiver
	\item Leica Robotic Total Station (RTS)
	\item Novatel SPAN IMU sensor
	\item Leica Nivel 210 inclinometer
\end{itemize}

\end{frame}

\begin{frame}{Plan}%{Daily repetition}
\begin{itemize}
	\item What you want to do?
	\item How you want to do it?
	\item How many ppl you have?
	\item Who is responsible for what?
	\item How can we prevent mistakes? How can we prevent accidents?
\end{itemize}
\end{frame}

\begin{frame}{H\&S}%{Daily repetition}
	%\centering
	\includegraphics[width=.7\textwidth]{pic/rainbow.jpg}
	\captionof{figure}{Have you got risk assessment?} %to make images on top of each other
\end{frame}


\section{Implementation}

\begin{frame}{Bridge layout}%{Daily repetition}
	\centering
	\includegraphics[height=.6\textheight]{pic/BridgeBare.jpg}
	\captionof{figure}{Top view} %to make images on top of each other
\end{frame}

\begin{frame}{Sensors layout}%{Daily repetition}
	\centering
	\includegraphics[height=.7\textheight]{pic/layout.jpg}
	\captionof{figure}{Location of the sensors} %to make images on top of each other

\end{frame}

\begin{frame}{Excitation}%{Daily repetition}
	To better understand sensors performance and to characterise bridge movement we will also create artificial excitation of the bridge.

	\centering
	\includegraphics[height=.6\textheight]{pic/excitation.jpg}
	\captionof{figure}{Excitation example from last year} %to make images on top of each other
	%\captionof{figure}{Example of visualisation}
\end{frame}


\section{Summary}


	\begin{frame}{Summary}
	%For older versions please refer to the GitHub site.
	\end{frame}


%Final slide
\setbeamercolor{background canvas}{bg=blueBgd!60}
\plain{Good luck}

\end{document}
