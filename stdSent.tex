%SHORTHANDS
%\newcommand*{\Myrange}[3]{$\textrm{\SIrange{#1}{#2}{#3}}$}

%std sentences
\newcommand*{\RefLoc}[1]{Section \vref{#1} provide more information.\xspace}
\newcommand*{\refLoc}[1]{as explained in section \vref{#1}.\xspace}

%HOW 2 reference Wikipedia Common Imaages and others
\newcommand*{\wiki}[1]{ \FigFootnote{Adapted after {#1} / CC BY-SA 3.0}}
\newcommand*{\wikiORG}[1]{ \FigFootnote{Work found at {#1} / CC BY-SA 3.0}}


%defos for GNSS/IMU
\newcommand*{\Q}{symmetric covariance-variance matrix Q\xspace}
\newcommand*{\sday}{Sidereal day is equivalent to $\sdayVal$.\xspace}
\newcommand*{\sdayVal}{$\SI{23.93447}{\hour}$\xspace}
\newcommand*{\IGS}{IGS orbit products\xspace}
\newcommand*{\Vacc}{Vertical accuracy is usually $\textrm{\numrange{1.2}{2.0}}$ times worse then the planar one.
\newcommand*{\amb}{the carrier-cycle integer ambiguity\xspace}

%GRID
\newcommand*{\LocalGrid}{Coordinates in the local grid (ENH) using arbitrary conversion and without applying the scale factor.\xspace}
\newcommand*{\OSGB}{Coordinates are in the OSGB grid (ENH), with scale factor applied.\xspace}

