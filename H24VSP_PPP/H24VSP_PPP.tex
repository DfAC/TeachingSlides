\documentclass[11pt]{beamer}

%my colours


\usetheme{metropolis}
\metroset{everytitleformat = regular,progressbar=foot} %settings
\mode<presentation>
%\usecolortheme{dove} %dove
% albatross, beaver, beetle, crane, default, dolphin, dove, orchid, rose, seagull, seahorse, whale, wolverine
%dont use  fly, lily,
%http://mirror.ox.ac.uk/sites/ctan.org/macros/latex/contrib/beamer/doc/beameruserguide.pdf
\setbeamercolor{title separator}{fg = UniBlue}
\setbeamercolor{frametitle}{fg = deepBlue, bg=aBlue!70}

\usepackage{booktabs}
\usepackage[scale=2]{ccicons}

\usepackage{pgfplots}
\usepgfplotslibrary{dateplot}


%coords are in relation to lower right corner
%\logo{\pgfputat{\pgfxy(0,6}{\pgfbox[right,top]{\includegraphics[width=2.5cm]{logo.png}}}}

\usepackage{tikz}
\addtobeamertemplate{frametitle}{}{%
\begin{tikzpicture}[remember picture,overlay]
\node[anchor=north east,yshift=2pt] at (current page.north east) {\includegraphics[height=0.8cm]{pic/logo.png}};
%\node[anchor=north east,yshift=2pt] at (current page.north east) {\includegraphics[height=0.8cm]{./../logo.png}};
\end{tikzpicture}}

%My std preamble for the docs
%\selectlanguage{british}%
\usepackage[british]{babel}
\usepackage{microtype} %better text
\IfFileExists{lmodern.sty}{\usepackage{lmodern}}{} %type 1 vector font
%
\usepackage{lettrine}
\usepackage{listings} %Add list support
\usepackage{colortbl} %colors in TABLES
%\usepackage{tikz,amsmath, amssymb,bm,color}
\usepackage{nicefrac}
\usepackage{lastpage} %get last page

%COLORS
\usepackage{color}
\definecolor{lightgray}{gray}{0.8} %for colortbl


%%%%%%%%%%%%%%%%%%%%%%%%%%%%%%%%%%%%%%%%%%%%%%%%%%%%%%%%%%%%%%%%%%
%FOOTNOTES
%nice look after http://www.dedoimedo.com
\usepackage[bottom,hang,splitrule]{footmisc} %\usepackage[bottom]{footmisc}
\setlength{\footnotemargin}{0.35cm} %space between number and footnote text
\setlength{\footnotesep}{0.2cm} %{12pt} %space between footnotes
\setlength{\skip\footins}{0.3cm} % space between the text body and the footnotes
%\setlength{\footskip}{0pt} dont do it. It will overflow page no

%%%%%%%%%%%%%%%%%%%%%%%%%%%%%%%%%%%%%%%%%%%%%%%%%%%%%%%%%%%%%%%%%%%
%TABLE SETTINGS
\usepackage{colortbl} %colors in table
\usepackage{rotating} %rotatins within tables
\usepackage{multirow}
\renewcommand{\arraystretch}{1.2} %add padding/spacing
%\usepackage{adjustbox}%rotating and fitting into page
%

\usepackage{booktabs} % To thicken table lines
%define thickness of table lines
\let\mytoprule\toprule
\renewcommand{\toprule}{\mytoprule[0.20em]}
\let\mytoprule\bottomrule
\renewcommand{\bottomrule}{\mytoprule[0.20em]}
\let\mytoprule\midrule
\renewcommand{\midrule}{\mytoprule[0.08em]}

\usepackage{spreadtab} % for simple calculations

%vertically and horizontally centered multicolumn cells with a fixed width. M{width}
%\newcolumntype{M}[1]{>{\centering\hspace{0pt}}m{#1}}
% each spanned cell has the same width. S{width of multicolumn cell}{number of spanned columns}
%\newcolumntype{S}[2]{>{\centering\hspace{0pt}}m{(#1+(2\tabcolsep+\arrayrulewidth)*(1-#2))/#2}}

%%%%%%%%%%%%%%%%%%%%%%%%%%%%%%%%%%%%%%%%%%%%%%%%%%%%%%%%%%%%%%%%%%%
%TikZ
\usepackage{tikz}

\colorlet{red}{red!50}
\colorlet{green}{green!50}
\colorlet{blue}{blue!50}
\definecolor{yellow}{HTML}{FFFF00}
\colorlet{yellow}{yellow!50}
\definecolor{fiolet}{HTML}{7030A0}
\colorlet{fiolet}{fiolet!50}
\colorlet{bgd_main}{black!50}
\colorlet{bgd}{bgd_main!75}
\colorlet{bgd2}{bgd_main!50}
\colorlet{bgd3}{bgd_main!25}
\colorlet{bgd4}{bgd_main!15}

\usetikzlibrary{shapes,arrows,calc,positioning}

%%%%%%%%%%%%%%%%%%%%%%%%%%%%%%%%%%%%%%%%%%%%%%%%%%%%%%%%%%%%%%%%%%%
% Footnotes in Figs
%\rule[raise-height]{width}{thickness}
\newcommand*{\FigFootnote}[1]{
\noindent \begin{flushleft}
\rule[0.2ex]{0.4\columnwidth}{0.5pt}
\par
\footnotesize
#1
\footnotesize
\end{flushleft}
}

%%%%%%%%%%%%%%%%%%%%%%%%%%%%%%%%%%%%%%%%%%%%%%%%%%%%%%%%%%%%%%%%%%%
%MATH, number display
%need to install siunitx, l3kernel,l3packages
\usepackage{siunitx} %this is for units display

\sisetup{per-mode=fraction, tight-spacing = true , fraction-function = \nicefrac, quotient-mode = fraction}% %nicefrac \tfrac
\sisetup{inter-unit-product = \ensuremath { { } \cdot { } } , exponent-product = \cdot }%
\sisetup{input-product=x , output-quotient =  \ensuremath { { } \times{}}} %for 1x2x3
%number grouping(3), std==true %\sisetup{group-digits = decimal} 
\sisetup{group-minimum-digits = 4} %start grouping from 4 digits, in 3 no groups
\sisetup{range-units = single,range-phrase = \,--\,} %2-3C not 2C-3C %, range-phrase = --
\sisetup{separate-uncertainty=true} %2+-1 not 2(1)
\sisetup{prefixes-as-symbols=true } % , scientific-notation = engineering false for 10^-9 ect ect , exp in multiple of 3
\sisetup{range-phrase = \,-\, } % , refo of ranges
%\sisetup{zero-decimal-to-integer, round-mode = places,round-precision = 3}
%\sisetup{add-arc-degree-zero=true , add-arc-minute-zero=true ,add-arc-second-zero=true} %for angle settings

%This is to auto convert ns,ms,us to 10^-xx s
\DeclareSIUnit[scientific-notation = engineering, prefixes-as-symbols=false]{\psec}{\pico\second}
\DeclareSIUnit[scientific-notation = engineering, prefixes-as-symbols=false]{\nsec}{\nano\second} 
\DeclareSIUnit[scientific-notation = engineering, prefixes-as-symbols=false]{\usec}{\micro\second}
\DeclareSIUnit[scientific-notation = engineering, prefixes-as-symbols=false]{\msec}{\milli\second} 

%...AND SOME UNITS
\DeclareSIUnit\dBm{dBm}
\DeclareSIUnit\ppm{ppm} %{\num{1e-6}}%{ppm}
\DeclareSIUnit\yr{yr} %{{361}\day}<-nice nice
\DeclareSIUnit\cy{cycle} %phase cycle
\DeclareSIUnit\epoch{epoch} %GPS/LL epoch
\DeclareSIUnit\inch{"} %inch 
\DeclareSIUnit\wk{week} %week
\DeclareSIUnit\hr{hrs} %hours
\DeclareSIUnit\min{minute} %hours
\DeclareSIUnit\mile{mi}
\DeclareSIUnit\Mcps{Mcps}
\DeclareSIUnit\bit{bit}
\DeclareSIUnit\chip{chip length}
%Other units
\newcommand*{\GBP}[1]{$\SI{#1}[\textsterling]{}$}


%%%%%SHORTHANDS (Standard Sentences)
%\newcommand*{\Myrange}[3]{$\textrm{\SIrange{#1}{#2}{#3}}$}

%details of bar tracking name
\newcommand*{\bsSys}{\textit{PASS Locator booking system}\xspace}


%how much work per week
\newcommand*{\wkWrk}[1]{$\SI{#1}{\hr\per\wk}$}

%references
\newcommand*{\tabref}[1]{shown in table \ref{#1} on page \pageref{#1}\xspace}
\newcommand*{\vref}[1]{\ref{#1} on page \pageref{#1}\xspace}


%\renewcommand{\baselinestretch}{1.2} %line spacing
%{\setstretch{1.0}\color{blue} text bla bla } for section strech
\renewcommand{\footnotesize}{\scriptsize}
%\usepackage[demo]{graphicx}
\usepackage[font={small,it}]{caption}
%\usepackage{subcaption}

%%%%%%%%%%%%%%%%%%%%%%%%%%%%%%%%%%%%%%%%%%%%%%%%%%%%%%%%%%%%%%%%%%%%%%%%%%%%%%

\title[H24VSP]{H24VSP Project 3}
\subtitle{Introduction to PPP using DL5}
\author{Lukasz K Bonenberg}
\institute{NGI}
\date{\today}
% \titlegraphic{\hfill\includegraphics[height=1.5cm]{logo/logo}}

\begin{document}

\maketitle

\begin{frame}{Table of Contents}

  \setbeamertemplate{section in toc}[sections numbered]
  \tableofcontents[hideallsubsections]
\end{frame}


%%%%%%%%%%%%%%%%%%%%%%%%%%%%%%%%%%%%%%%%%%%%%%%%%%%%%%%%%%%%%%%%%%%%%%%%%%%%%%%%%%%%%%%%
\section{Introduction}

\begin{frame}[allowframebreaks=0.8]]{Introduction}
%\begin{frame}[fragile]{Introduction}

	Aim of the H24VSP module is to develop the practical understanding of the practical aspects of
satellite-based positioning and its applications. In the last practical we will explore the capacities of the Precise Point Positoning (PPP) by comparing it with Network RTK that you are allready familiar with. We are interested in assessing difference between:
	\begin{itemize}
		\item convergence time;
		\item precision - estimated and actual after convergence; 
		\item accuracy after convergence.
	\end{itemize}
	
	We will be using:
	\begin{itemize}
		\item Leica GS10;
		\item maritime Veripos LD5 receiver with AsterRx chipset\footnote[frame]{For short introductory video see \url{http://bit.ly/VeriposLD5}.}.
	\end{itemize}
%\href{https://github.com/hsrmbeamertheme/hsrmbeamertheme}{\textsc{hsrm} Beamer Theme} by Benjamin Weiss.
\end{frame}

%%%%%%%%%%%%%%%%%%%%%%%%%%%%%%%%%%%%%%%%%%%%%%%%%%%%%%%%%%%%%%%%%%%%%%%%%%%%%%%%%%%%%%%%
\section{Veripos Services}

\begin{frame}[allowframebreaks]{Veripos Services}
	Veripos offer hardware (receivers) in combination with following services\footnote[frame]{Also check \url{http://www.veripos.com/services.html} and for video \url{http://bit.ly/VeriposServices}.}:
	
	\begin{itemize}
		\item \textbf{Veripos Standard} with single frequency DGPS and 1-2 metre accuracy.
		\item {\color{gray}\textbf{Legacy Veripos Standard Plus} with dual-frequency DGPS for low latitude areas and 1-2 metre accuracy.} 
		\item \textbf{Veripos Standard$^2$ } with single frequency combined GPS and GLONASS DGPS. 
		\item \textbf{Veripos Ultra/APEX} using global orbit, clock correction and dual-frequency GPS/GLONASS observations for dm level accuracy.
	\end{itemize}
	Corrections are transmitted via Inmarsat geostationary satellites - 25E, 98W, 143.5E, AORE, AORW, IOR, POR\footnote[frame]{\url{http://www.veripos.com/global-coverage.html}}. All coordinates provided are in ITRF2008.
\end{frame}

\begin{frame}{Veripos Standard}
	
	\begin{itemize}	
		\item Provides RTCM Type 1\footnote{DGPS corrections.}, 3\footnote{GPS reference station parameters.} messages.
		\item Normal accuracy: 1-2m. 
		\item Typical latency: 4 seconds\footnote{Typical correction update interval is 15 seconds.}.
		\item Single difference (DGPS) using GPS C/A code
	\end{itemize}	
	
\end{frame}

\begin{frame}{Veripos Standard Plus}
	Standard Plus is intended to support DGPS positioning for lower latitudes and combat ionospheric activity. 
	\begin{itemize}	
		\item Provides RTCM Type 1, 3, 15\footnote[frame]{Ground transmitter parameters including ionospheric delay information} messages.
		\item Normal accuracy: 1-2m. 
		\item Typical latency: 4 seconds.
		\item Single difference (DGPS) using GPS C/A and P code
	\end{itemize}	
	Note that this is legacy service, according to \url{http://www.veripos.com/services.html}.
\end{frame}

\begin{frame}{Veripos Standard and Standard Plus}
	\begin{figure}
		\includegraphics[height=0.8\textheight]{pic/StdPlus.png}
		\caption{Solutions at a monitor site in Malongo [Veripos]}
	\end{figure}
	
\end{frame}


\begin{frame}{Veripos Standard$^2$}
	
	\begin{itemize}	
		\item Provides RTCM Type 1, 3, 31\footnote{DGPS GLONASS corrections.}, 32\footnote{GPS GLONASS reference station parameters.} messages.
		\item Normal accuracy: 1-2m. 
		\item Typical latency: 4 seconds.
		\item Single difference (DGPS) using GPS and GLONASS C/A code\footnote[frame]{It is possible to use GLONASS only with this service as well.}
	\end{itemize}	
	
\end{frame}

\begin{frame}{Veripos Ultra}
	\begin{columns}[T,onlytextwidth]
		\column{0.5\textwidth}
		\begin{itemize}	
			\item Orbit and clock corrections in JPL GDGPS format.
			\item Normal accuracy: 0.1m planar. 
			\item Typical latency: 2 seconds with 30 s update rate.% for clocks and 120s for orbits.
			\item Precise Point Positioning (PPP) using C/A and P code and L1/L2 carrier phase for GPS.
		\end{itemize}	
		\column{0.5\textwidth}
		\begin{figure}[T]
			\vspace*{-1cm}
			\includegraphics[height=0.7\textheight]{pic/Ultra.png}
			\caption{Standard and Ultra solutions at a monitor site in Singapore.}
		\end{figure}
	\end{columns}
\end{frame}


\begin{frame}{Veripos Apex$^2$}
	\begin{columns}[T,onlytextwidth]
		\column{0.5\textwidth}
		\begin{itemize}	
		\item Orbit and clock corrections in Veripos OCDS format.
		\item Normal accuracy: 0.1m planar. 
		\item Typical latency: 2 seconds with 30 s update rate.% for clocks and 120s for orbits.
		\item Precise Point Positioning (PPP) using C/A and P code and L1/L2 carrier phase for GPS and GLONASS.
		\end{itemize}	
		\column{0.5\textwidth}
		\begin{figure}[T]
			\vspace*{-1cm}
			\includegraphics[height=0.8\textheight]{pic/Apex.png}
			\caption{Veripos Apex solution at a monitor site in Aberdeen.}
		\end{figure}
	\end{columns}
\end{frame}


%%%%%%%%%%%%%%%%%%%%%%%%%%%%%%%%%%%%%%%%%%%%%%%%%%%%%%%%%%%%%%%%%%%%%%%%%%%%%%%%%%%%%%%%%%%%%%%%%%%%%%
\section{Live demo}



%%%%%%%%%%%%%%%%%%%%%%%%%%%%%%%%%%%%%%%%%%%%%%%%%%%%%%%%%%%%%%%%%%%%%%%%%%%%%%%%%%%%%%%%%%%%%%%%%%%%%%
\section{Practical work}

\begin{frame}{Practice layout}
	
	\begin{itemize}
		\item LD5 will be restarted at 12:00 in order to converge properly.
		\item You will start collecting RTK data after 14:00. 
		\item You will download Veripos NMEA strings for Ultra and Apex$^2$ alongside data from GS10.
		\item Make sure that Veripos NMEA file has been splitted into \$GPGGA and \$GPGST before leaving.
	\end{itemize}

\end{frame}


\begin{frame}{NGB5}
	\begin{table}
		\centering
		\begin{minipage}[t]{\textwidth}%
			\resizebox{\columnwidth}{!}{%
				\begin{tabular}{lccccc} %{l|c|c|c|c}
					\toprule %\rowcolor{lightgray}
					Point & Frame & Lat[deg]  & Long[deg]  & EllHt[m] & Notes\tabularnewline
					\midrule
					NGB5 & ETRF97 & 52 57 07.05318 & 01 11 01.44897 W & 91.2065 & at point\\
					NGB5 & ETRF97 & 52 57 07.05318 & 01 11 01.44897 W & 91.3865 & at ARP\footnote{Antenna heigh = 0.18m.}\\
					NGB5 & ETRF97 & 52 57 07.05318 & 01 11 01.44897 W & 91.4280 & at antenna PCO\footnote{Antenna offset for ionsphere free solution is $2.545L_1-1.545L_2$ so\\ $2.545*55.3-1.545*64.2=41.5mm$.}\\
					\textbf{NGB5} & ITRF2008 & \textbf{52 57 7.070524} & \textbf{01 11 1.427085} W & \textbf{91.480} & at antenna PCO\footnote{Converted from ETRF97 to ITRF2008 at epoch 2015-12-04.} \\
					\textbf{NGB5} & ITRF2008 & \textbf{5257.1178421} & \textbf{00111.0237848} W & \textbf{91.480} & at antenna PCO\footnote{Converted to DDMM.MMMMMMM to be compatible with NEMEA GGA string.} \\
					
					\bottomrule
				\end{tabular}%
			}
			\caption{Coordinates of NGB5}
		\end{minipage}
	\end{table}
	%in DDMM.MMMMMMM 7 decimal place is eq to 0.0018 m	
\end{frame}



\begin{frame}{Veripos \$GPGGA NMEA strings}
	
	In Verpos provides two types of NMEA strings \$GPGGA and \$GPGST. \$GPGGA will behave differently in PPP mode with QA flag always 2 or 5. To obtain any information about solution we need to examine last flag before CRC(*).\\
	
	
	\begin{exampleblock}{Example}
		{\tiny{\$GPGGA,183324.00,5257.1178371,N,00111.0236798,W,\textbf{5},17,0.7,42.76,M,49.01,M,30.5,{\color{red}\textbf{0268}}*54}.}
	\end{exampleblock}
	
	Values for the flag indicate:	
	\begin{description}
		\item[0268] $ULTRA^2$
		\item [0281]	$APEX^2$
		\item [0068]	ULTRA
		\item [0081]	APEX
		\item [1006]	$Standard^2$
	\end{description}
	%use  cut -d, -f15 *.txt | cut -d* -f1 | sort | uniq to extract data  
\end{frame}

\begin{frame}[plain]{Veripos \$GPGST NMEA strings}

	\begin{exampleblock}{Example}
		{\textit{\$GPGST,140545.00,3.81,0.02,0.01,81.00,0.02,0.01,0.02*57}.}
	\end{exampleblock}
	\vspace*{-1cm}
	\begin{table}
		\centering
		\begin{minipage}[t]{\textheight}%
			\resizebox{\columnwidth}{!}{%
				\begin{tabular}{lc} %{l|c|c|c|c}
					\toprule %\rowcolor{lightgray}
					Cell & Notes\tabularnewline
					\midrule
					0&Message ID \$GPGST\\
					1&UTC of position fix\footnote{Notice 17s offset to GPS time.}\\
					2&RMS value of the pseudorange or carrier phase (RTK/PPP) residuals\\
					3&Error ellipse semi-major axis 1 sigma error, in meters\\
					4&Error ellipse semi-minor axis 1 sigma error, in meters\\
					5&Error ellipse orientation, degrees from true north\\
					6&Latitude 1 sigma error, in meters\\
					7&Longitude 1 sigma error, in meters\\
					8&Height 1 sigma error, in meters\\
					9&The checksum data, always begins with *\\
					\bottomrule
				\end{tabular}%
			}
			%	\caption{Coordinates of NGB5}
		\end{minipage}
	\end{table}
	
	
	
\end{frame}

%Final slide
\setbeamercolor{background canvas}{bg=blueBgd!60}
\plain{Questions?}



\end{document}
