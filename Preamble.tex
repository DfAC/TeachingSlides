%My std preamble for the docs
%\selectlanguage{british}%
\usepackage[british]{babel}
\usepackage{microtype} %better text
\IfFileExists{lmodern.sty}{\usepackage{lmodern}}{} %type 1 vector font
%
\usepackage{lettrine}
\usepackage{listings} %Add list support
\usepackage{colortbl} %colors in TABLES
%\usepackage{tikz,amsmath, amssymb,bm,color}
\usepackage{nicefrac}
\usepackage{lastpage} %get last page

%COLORS
\usepackage{color}
\definecolor{lightgray}{gray}{0.8} %for colortbl


%%%%%%%%%%%%%%%%%%%%%%%%%%%%%%%%%%%%%%%%%%%%%%%%%%%%%%%%%%%%%%%%%%
%FOOTNOTES
%nice look after http://www.dedoimedo.com
\usepackage[bottom,hang,splitrule]{footmisc} %\usepackage[bottom]{footmisc}
\setlength{\footnotemargin}{0.35cm} %space between number and footnote text
\setlength{\footnotesep}{0.2cm} %{12pt} %space between footnotes
\setlength{\skip\footins}{0.3cm} % space between the text body and the footnotes
%\setlength{\footskip}{0pt} dont do it. It will overflow page no

%%%%%%%%%%%%%%%%%%%%%%%%%%%%%%%%%%%%%%%%%%%%%%%%%%%%%%%%%%%%%%%%%%%
%TABLE SETTINGS
\usepackage{colortbl} %colors in table
\usepackage{rotating} %rotatins within tables
\usepackage{multirow}
\renewcommand{\arraystretch}{1.2} %add padding/spacing
%\usepackage{adjustbox}%rotating and fitting into page
%

\usepackage{booktabs} % To thicken table lines
%define thickness of table lines
\let\mytoprule\toprule
\renewcommand{\toprule}{\mytoprule[0.20em]}
\let\mytoprule\bottomrule
\renewcommand{\bottomrule}{\mytoprule[0.20em]}
\let\mytoprule\midrule
\renewcommand{\midrule}{\mytoprule[0.08em]}

\usepackage{spreadtab} % for simple calculations

%vertically and horizontally centered multicolumn cells with a fixed width. M{width}
%\newcolumntype{M}[1]{>{\centering\hspace{0pt}}m{#1}}
% each spanned cell has the same width. S{width of multicolumn cell}{number of spanned columns}
%\newcolumntype{S}[2]{>{\centering\hspace{0pt}}m{(#1+(2\tabcolsep+\arrayrulewidth)*(1-#2))/#2}}

%%%%%%%%%%%%%%%%%%%%%%%%%%%%%%%%%%%%%%%%%%%%%%%%%%%%%%%%%%%%%%%%%%%
%TikZ
\usepackage{tikz}

\colorlet{red}{red!50}
\colorlet{green}{green!50}
\colorlet{blue}{blue!50}
\definecolor{yellow}{HTML}{FFFF00}
\colorlet{yellow}{yellow!50}
\definecolor{fiolet}{HTML}{7030A0}
\colorlet{fiolet}{fiolet!50}
\colorlet{bgd_main}{black!50}
\colorlet{bgd}{bgd_main!75}
\colorlet{bgd2}{bgd_main!50}
\colorlet{bgd3}{bgd_main!25}
\colorlet{bgd4}{bgd_main!15}

\usetikzlibrary{shapes,arrows,calc,positioning}

%%%%%%%%%%%%%%%%%%%%%%%%%%%%%%%%%%%%%%%%%%%%%%%%%%%%%%%%%%%%%%%%%%%
% Footnotes in Figs
%\rule[raise-height]{width}{thickness}
\newcommand*{\FigFootnote}[1]{
\noindent \begin{flushleft}
\rule[0.2ex]{0.4\columnwidth}{0.5pt}
\par
\footnotesize
#1
\footnotesize
\end{flushleft}
}

%%%%%%%%%%%%%%%%%%%%%%%%%%%%%%%%%%%%%%%%%%%%%%%%%%%%%%%%%%%%%%%%%%%
%MATH, number display
%need to install siunitx, l3kernel,l3packages
\usepackage{siunitx} %this is for units display

\sisetup{per-mode=fraction, tight-spacing = true , fraction-function = \nicefrac, quotient-mode = fraction}% %nicefrac \tfrac
\sisetup{inter-unit-product = \ensuremath { { } \cdot { } } , exponent-product = \cdot }%
\sisetup{input-product=x , output-quotient =  \ensuremath { { } \times{}}} %for 1x2x3
%number grouping(3), std==true %\sisetup{group-digits = decimal} 
\sisetup{group-minimum-digits = 4} %start grouping from 4 digits, in 3 no groups
\sisetup{range-units = single,range-phrase = \,--\,} %2-3C not 2C-3C %, range-phrase = --
\sisetup{separate-uncertainty=true} %2+-1 not 2(1)
\sisetup{prefixes-as-symbols=true } % , scientific-notation = engineering false for 10^-9 ect ect , exp in multiple of 3
\sisetup{range-phrase = \,-\, } % , refo of ranges
%\sisetup{zero-decimal-to-integer, round-mode = places,round-precision = 3}
%\sisetup{add-arc-degree-zero=true , add-arc-minute-zero=true ,add-arc-second-zero=true} %for angle settings

%This is to auto convert ns,ms,us to 10^-xx s
\DeclareSIUnit[scientific-notation = engineering, prefixes-as-symbols=false]{\psec}{\pico\second}
\DeclareSIUnit[scientific-notation = engineering, prefixes-as-symbols=false]{\nsec}{\nano\second} 
\DeclareSIUnit[scientific-notation = engineering, prefixes-as-symbols=false]{\usec}{\micro\second}
\DeclareSIUnit[scientific-notation = engineering, prefixes-as-symbols=false]{\msec}{\milli\second} 

%...AND SOME UNITS
\DeclareSIUnit\dBm{dBm}
\DeclareSIUnit\ppm{ppm} %{\num{1e-6}}%{ppm}
\DeclareSIUnit\yr{yr} %{{361}\day}<-nice nice
\DeclareSIUnit\cy{cycle} %phase cycle
\DeclareSIUnit\epoch{epoch} %GPS/LL epoch
\DeclareSIUnit\inch{"} %inch 
\DeclareSIUnit\wk{week} %week
\DeclareSIUnit\hr{hrs} %hours
\DeclareSIUnit\min{minute} %hours
\DeclareSIUnit\mile{mi}
\DeclareSIUnit\Mcps{Mcps}
\DeclareSIUnit\bit{bit}
\DeclareSIUnit\chip{chip length}
%Other units
\newcommand*{\GBP}[1]{$\SI{#1}[\textsterling]{}$}


%%%%%SHORTHANDS (Standard Sentences)
%\newcommand*{\Myrange}[3]{$\textrm{\SIrange{#1}{#2}{#3}}$}

%details of bar tracking name
\newcommand*{\bsSys}{\textit{PASS Locator booking system}\xspace}


%how much work per week
\newcommand*{\wkWrk}[1]{$\SI{#1}{\hr\per\wk}$}

%references
\newcommand*{\tabref}[1]{shown in table \ref{#1} on page \pageref{#1}\xspace}
\newcommand*{\vref}[1]{\ref{#1} on page \pageref{#1}\xspace}

