% support for my latex code
% LKB (c) 2016
%%%%%%%%%%%%%%%%%%%%%%%%%%%%%%%%%%%%%%%%%%%%%%%%%%%%%%%%%%%%%%%%%%%%%%%%
%NOTE boxes
\usepackage[most]{tcolorbox}
%\usepackage{lipsum}

\newcommand*{\NOTE}[1]{
	\begin{tcolorbox}[colback=blue!15!,colframe=blue!60!black,fonttitle=\bfseries,
		colbacktitle=blue!60!black,enhanced,
		attach boxed title to top center={yshift=-2mm},title=NOTES]
		{#1}
	\end{tcolorbox}
}

%%%%%%%%%%%%%%%%%%%%%%%%%%%%%%%%%%%%%%%%%%%%%%%%%%%%%%%%%%%%%%%%%%%%%%%%
%TABLE SETTINGS
\usepackage{colortbl} %colors in table
\usepackage{rotating} %rotatins within tables
\usepackage{multirow}
\renewcommand{\arraystretch}{1.2} %add padding/spacing
%\usepackage{adjustbox}%rotating and fitting into page

%define thickness of table lines
\let\mytoprule\toprule
\renewcommand{\toprule}{\mytoprule[0.20em]}
\let\mytoprule\bottomrule
\renewcommand{\bottomrule}{\mytoprule[0.20em]}
\let\mytoprule\midrule
\renewcommand{\midrule}{\mytoprule[0.08em]}


%%%%%%%%%%%%%%%%%%%%%%%%%%%%%%%%%%%%%%%%%%%%%%%%%%%%%%%%%%%%%%%%%%%%%%%%
%COLORS
\usepackage{color}
%\definecolor{lightgray}{gray}{0.1} %for colortbl
%\definecolor{lightgray}{rgb}{0.2,0.2,0.2}
\definecolor{verylight}{gray}{0.05}
\definecolor{mygray}{rgb}{0.5,0.5,0.5}
%\definecolor{uniBlue}{rgb}{0,0.1,0.3}
\definecolor{uniBlueBgd}{RGB}{30,210,255}
\definecolor{uniBlue}{RGB}{0,128,255}

%%%%%%%%%%%%%%%%%%%%%%%%%%%%%%%%%%%%%%%%%%%%%%%%%%%%%%%%%%%%%%%%%%%%%%%%
%LISTS
%\setitemize[0]{leftmargin=15pt,itemindent=0pt,labelwidth=10pt}
\usepackage{enumitem}
%\setlist{nosep} % or \setlist{noitemsep} to leave space around whole list
%\begin{enumerate}[itemsep=1pt, topsep=12pt, partopsep=0pt]
\setlist{noitemsep, topsep=10pt, partopsep=0pt}

%%%%%%%%%%%%%%%%%%%%%%%%%%%%%%%%%%%%%%%%%%%%%%%%%%%%%%%%%%%%%%%%%%%
%MATH, number display
%need to install siunitx, l3kernel,l3packages
\usepackage{siunitx} %this is for units display

\sisetup{per-mode=fraction, tight-spacing = true , fraction-function = \nicefrac, quotient-mode = fraction}% %nicefrac \tfrac
\sisetup{inter-unit-product = \ensuremath { { } \cdot { } } , exponent-product = \cdot }%
\sisetup{input-product=x , output-quotient =  \ensuremath { { } \times{}}} %for 1x2x3
%number grouping(3), std==true %\sisetup{group-digits = decimal} 
\sisetup{group-minimum-digits = 4} %start grouping from 4 digits, in 3 no groups
\sisetup{range-units = single,range-phrase = \,--\,} %2-3C not 2C-3C %, range-phrase = --
\sisetup{separate-uncertainty=true} %2+-1 not 2(1)
\sisetup{prefixes-as-symbols=true } % , scientific-notation = engineering false for 10^-9 ect ect , exp in multiple of 3
%\sisetup{zero-decimal-to-integer, round-mode = places,round-precision = 3}
%\sisetup{add-arc-degree-zero=true , add-arc-minute-zero=true ,add-arc-second-zero=true} %for angle settings

%This is to auto convert ns,ms,us to 10^-xx s
\DeclareSIUnit[scientific-notation = engineering, prefixes-as-symbols=false]{\psec}{\pico\second}
\DeclareSIUnit[scientific-notation = engineering, prefixes-as-symbols=false]{\nsec}{\nano\second} 
\DeclareSIUnit[scientific-notation = engineering, prefixes-as-symbols=false]{\usec}{\micro\second}
\DeclareSIUnit[scientific-notation = engineering, prefixes-as-symbols=false]{\msec}{\milli\second} 

%...AND SOME UNITS
\DeclareSIUnit\dBm{dBm}
\DeclareSIUnit\ppm{ppm} %{\num{1e-6}}%{ppm}
\DeclareSIUnit\yr{yr} %{{361}\day}<-nice nice
\DeclareSIUnit\cy{cycle} %phase cycle
\DeclareSIUnit\epoch{epoch} %GPS/LL epoch
\DeclareSIUnit\inch{"} %inch 
\DeclareSIUnit\wk{GPS week} %week
\DeclareSIUnit\mile{mi}
\DeclareSIUnit\Mcps{Mcps}
\DeclareSIUnit\bit{bit}

%%%%%%%%%%%%%%%%%%%%%%%%%%%%%%%%%%%%%%%%%%%%%%%%%%%%%%%%%%%%%%%%%%%%%%%%
%CAPTION
%\usepackage{caption} %caps
%\captionsetup{labelfont=bf,format=plain,indention=0cm,font=small,justification=raggedright,singlelinecheck=false,labelsep=endash} 

%%%%%%%%%%%%%%%%%%%%%%%%%%%%%%%%%%%%%%%%%%%%%%%%%%%%%%%%%%%%%%%%%%%%%%%%
%SETTINGS FOR THE TikZ
% \colorlet{red}{red!50}
% \colorlet{green}{green!50}
% \colorlet{blue}{blue!50}
% \definecolor{yellow}{HTML}{FFFF00}
% \colorlet{yellow}{yellow!50}
% \definecolor{fiolet}{HTML}{7030A0}
% \colorlet{fiolet}{fiolet!50}
% \colorlet{bgd_main}{black!50}
% \colorlet{bgd}{bgd_main!75}
% \colorlet{bgd2}{bgd_main!50}
% \colorlet{bgd3}{bgd_main!25}
% \colorlet{bgd4}{bgd_main!15}

% \usetikzlibrary{shapes,arrows,calc,positioning}


%%%%%%%%%%%%%%%%%%%%%%%%%%%%%%%%%%%%%%%%%%%%%%%%%%%%%%%%%%%%%%%%%%%%%%%%
%SHORTHANDS
%\newcommand*{\Myrange}[3]{$\textrm{\SIrange{#1}{#2}{#3}}$}
\newcommand*{\refLoc}[1]{as explained in section \ref{#1}\xspace}

%\newcommand*{\RefLoc}[1]{Section \vref{#1} provide more information.\xspace}





%HOW 2 reference Wikipedia Common Imaages and others
\newcommand*{\wiki}[1]{ \FigFootnote{Adapted after {#1} / CC BY-SA 3.0}}
\newcommand*{\wikiORG}[1]{ \FigFootnote{Work found at {#1} / CC BY-SA 3.0}}


%defos for GNSS/IMU
\newcommand*{\Q}{symmetric covariance-variance matrix Q\xspace}
\newcommand*{\sday}{Sidereal day is equivalent to $\sdayVal$.\xspace}
\newcommand*{\sdayVal}{$\SI{23.93447}{\hour}$\xspace}
\newcommand*{\IGS}{IGS orbit products\xspace}
\newcommand*{\amb}{the carrier-cycle integer ambiguity\xspace}

%GRID
\newcommand*{\LocalGrid}{Coordinates in the local grid (ENH) using arbitrary conversion and without applying the scale factor.\xspace}
\newcommand*{\OSGB}{Coordinates are in the OSGB grid (ENH), with scale factor applied.\xspace}

